\documentclass{amsart}

%\documentclass[10 pt]{amsart}

\usepackage[ocgcolorlinks,linktoc=all]{hyperref}
\hypersetup{citecolor=blue,linkcolor=red}
%\usepackage[parfill]{parskip}


%\usepackage{amsthm}
\usepackage{cleveref}
\crefname{lemma}{Lemma}{Lemmata}
\crefname{equation}{equation}{equations}

\newtheorem{theorem}{Theorem}
\newtheorem{lemma}[theorem]{Lemma}
\newtheorem{proposition}[theorem]{Proposition}
\newtheorem{corollary}[theorem]{Corollary}

\newtheorem*{thmA}{Theorem}
\newtheorem*{thmB}{Theorem}
\newtheorem*{rem}{Remark}
\newtheorem*{thmmain}{Theorem}
\newtheorem*{propmain}{Proposition}

\theoremstyle{definition}
\newtheorem{definition}[theorem]{Definition}
\newtheorem{example}[theorem]{Example}
\newtheorem{xca}[theorem]{Exercise}

\theoremstyle{remark}
\newtheorem{remark}[theorem]{Remark}

\numberwithin{equation}{section}

%Symbols
\renewcommand{\~}{\tilde}
\renewcommand{\-}{\bar}
\newcommand{\bs}{\backslash}
\newcommand{\cn}{\colon}
\newcommand{\sub}{\subset}

\newcommand{\N}{\mathbb{N}}
\newcommand{\R}{\mathbb{R}}
\newcommand{\Z}{\mathbb{Z}}
\renewcommand{\S}{\mathbb{S}}
\renewcommand{\H}{\mathbb{H}}
\newcommand{\C}{\mathbb{C}}
\newcommand{\K}{\mathbb{K}}
\newcommand{\Di}{\mathbb{D}}
\newcommand{\B}{\mathbb{B}}
\newcommand{\8}{\infty}

%Greek letters
\renewcommand{\a}{\alpha}
\renewcommand{\b}{\beta}
\newcommand{\g}{\gamma}
\renewcommand{\d}{\delta}
\newcommand{\e}{\epsilon}
\renewcommand{\k}{\kappa}
\renewcommand{\l}{\lambda}
\renewcommand{\o}{\omega}
\renewcommand{\t}{\theta}
\newcommand{\s}{\sigma}
\newcommand{\p}{\varphi}
\newcommand{\z}{\zeta}
\newcommand{\vt}{\vartheta}
\renewcommand{\O}{\Omega}
\newcommand{\D}{\Delta}
\newcommand{\G}{\Gamma}
\newcommand{\T}{\Theta}
\renewcommand{\L}{\Lambda}

%Mathematical operators
\newcommand{\INT}{\int_{\O}}
\newcommand{\DINT}{\int_{\d\O}}
\newcommand{\Int}{\int_{-\infty}^{\infty}}
\newcommand{\del}{\partial}

\newcommand{\inpr}[2]{\left\langle #1,#2 \right\rangle}
\newcommand{\fr}[2]{\frac{#1}{#2}}
\newcommand{\x}{\times}

\DeclareMathOperator{\dive}{div}
\DeclareMathOperator{\id}{id}
\DeclareMathOperator{\pr}{pr}
\DeclareMathOperator{\Diff}{Diff}
\DeclareMathOperator{\supp}{supp}
\DeclareMathOperator{\graph}{graph}
\DeclareMathOperator{\osc}{osc}
\DeclareMathOperator{\const}{const}
\DeclareMathOperator{\dist}{dist}
\DeclareMathOperator{\loc}{loc}

%Environments
\newcommand{\Theo}[3]{\begin{#1}\label{#2} #3 \end{#1}}
\newcommand{\pf}[1]{\begin{proof} #1 \end{proof}}
\newcommand{\eq}[1]{\begin{equation}\begin{alignedat}{2} #1 \end{alignedat}\end{equation}}
\newcommand{\IntEq}[4]{#1&#2#3	 &\quad &\text{in}~#4,}
\newcommand{\BEq}[4]{#1&#2#3	 &\quad &\text{on}~#4}
\newcommand{\br}[1]{\left(#1\right)}



%Logical symbols
\newcommand{\Ra}{\Rightarrow}
\newcommand{\ra}{\rightarrow}
\newcommand{\hra}{\hookrightarrow}
\newcommand{\mt}{\mapsto}

% Aleksandrov Reflection Macros
\DeclareMathOperator{\reflectionvector}{V}
\DeclareMathOperator{\reflectionangle}{\delta}
\newcommand{\reflectionplane}[1][\reflectionvector]{\ensuremath{P_{#1}}}
\newcommand{\reflectionmap}[1][\reflectionvector]{\ensuremath{R_{#1}}}
\newcommand{\reflectionset}[2][\reflectionvector]{\ensuremath{{#2}_{#1}}}
\newcommand{\reflectionhalfspace}[1][\reflectionvector]{\ensuremath{\reflectionset[{#1}]{H}}}
\DeclareMathOperator{\vertvec}{e}
\DeclareMathOperator{\origin}{O}
\DeclareMathOperator{\radialprojection}{\pi}
\DeclareMathOperator{\height}{h}
\DeclareMathOperator{\equator}{E}
\newcommand{\ip}[2]{\ensuremath{\langle{#1},{#2}\rangle}}
\DeclareMathOperator{\intersect}{\cap}
\DeclareMathOperator{\nor}{\nu}
\DeclareMathOperator{\basepoint}{p_0}
\DeclareMathOperator{\radialdistance}{r}

%Fonts
\newcommand{\mc}{\mathcal}
\renewcommand{\it}{\textit}
\newcommand{\mrm}{\mathrm}

%Spacing
\newcommand{\hp}{\hphantom}


\parindent 0 pt

\protected\def\ignorethis#1\endignorethis{}
\let\endignorethis\relax
\def\TOCstop{\addtocontents{toc}{\ignorethis}}
\def\TOCstart{\addtocontents{toc}{\endignorethis}}


\begin{document}
\title[Differential Harnack inequalities on the sphere]{Harnack inequalities for evolving hypersurfaces on the sphere}





\author[P. Bryan]{Paul Bryan}
\address{Leibniz Universit\"{a}t Hannover, Institut f\"{u}r Differentialgeometrie
und Riemann Center for Geometry and Physics, Welfengarten 1, 30167
Hannover, Germany}
\email{pabryan@gmail.com}
\author[M.N. Ivaki]{Mohammad N. Ivaki}
\address{Institut f\"{u}r Diskrete Mathematik und Geometrie, Technische Universit\"{a}t Wien,
Wiedner Hauptstr. 8--10, 1040 Wien, Austria}
\email{mohammad.ivaki@tuwien.ac.at}
\author[J. Scheuer]{Julian Scheuer }
\address{Albert-Ludwigs-Universit\"{a}t, 
Mathematisches Institut, Eckerstr. 1, 79104
Freiburg, Germany}
\email{julian.scheuer@math.uni-freiburg.de}
\date{\today}

\dedicatory{}
\subjclass[2010]{}
\keywords{Fully nonlinear curvature flows, Harnack estimates}


\begin{abstract}
We prove Harnack inequalities for hypersurfaces evolving on the unit sphere either by a 1-homogeneous convex curvature function or by the $p$-power of mean curvature with $0<p<1$. 
\end{abstract}

\maketitle


\section{Introduction}
We consider the evolution of a hypersurface $M^n$ by
\eq{\label{eq:CurvFlow}
\partial_tx = -\varphi\nu,~ x:M^n\times[0,T)\to M_c,
}
where \(M_c\) is the simply connected space form of constant sectional curvature \(c\), and $\p\in C^{\8}(\G_+)$ is a strictly monotone,  symmetric function on the eigenvalues of Weingarten map \(\mathcal{W}\) (principal curvatures) \(\kappa_1, \cdots, \kappa_n\). Strict monotonicity ensures the flow is parabolic. We will need to make some further assumptions on the speed to obtain Harnack inequalities. Either,
\begin{itemize}
\item \(\p\) is 1-homogeneous and convex, or
\item \(\p = H^p, 0 < p\leq 1\).
\end{itemize}

Under these assumptions, our principal results are Harnack inequalities for flows on the sphere. These results extend the Harnack inequalities obtained in \cite{IvakiBryan:08/2015, bryanlouie} on the sphere to a broader class of flows, similar to, though more restrictive than the class of flows in Euclidean space \cite{MR1296393} for which Harnack inequalities are known.

\begin{theorem}\label{thm:harnack}
Let $\p$ be a  convex curvature function. Then
\[
\partial_t \p-b^{ij}\nabla_i\p\nabla_j\p+\frac{1}{2}\frac{\p}{t}>0.
\]

Let $\p(H)=H^{p}.$ If $\frac{1}{2}+\frac{1}{2n}\leq {p}< 1,$ then
\[
\partial_t H^{p} - b^{ij}\nabla_iH^{p}\nabla_jH^{p} - \frac{c {p}}{2{p}-1}H^{2{p}-1} + \frac{{p}}{{p}+1} \frac{H^{p}}{t} > 0.
\]
If $0<{p}\leq \frac{1}{2} + \frac{1}{2n}$ or $p=1$, then
\[
\partial_t H^{p} - b^{ij}\nabla_iH^{p}\nabla_jH^{p} - c n{p}H^{2{p}-1} + \frac{{p}}{{p}+1} \frac{H^{p}}{t} > 0.
\]
\end{theorem}

A number of authors have studied Harnack inequalities in Euclidean space. The genesis of such study is \cite{MR1316556} where Hamilton proves a Harnack inequality for the mean curvature flow of convex hypersurfaces. Harnack inequalities for other flows have been obtained in \cite{MR1100812,ivaki2015centro,ivaki2015convex, MR2813400, MR1480081}, including flows by the power of Gauss curvature, centro-affine normal flows, and flows by the power of inverse mean curvature. The most general results, subsuming many other results, were obtained in \cite{MR1296393} for so-called \(\alpha\)-convex and \(\alpha\)-concave speeds.

In the sphere, we find it rather more difficult to obtain Harnack inequalities in such generality. For one thing, the Gauss-map parametrization used to great effect in \cite{MR1296393} is not available to us (at least in the same way). We get around this first difficulty by direct computation. The principle difficulty, however, arises from the background curvature introducing extra terms, involving up to third derivatives of the speed, and at this stage, we are only able to obtain the required positivity in the cases listed above. In the case of Euclidean space, our computations do recover many (but no new) Euclidean Harnack inequalities about which we make brief mention in \cref{sec:Harnack}.

Let us now recall the general philosophy that ``equality should be attained on solitons'' in Harnack inequalities for curvature flows as espoused in \cite{MR1316556}, and see how this philosophy may be interpreted in the sphere. The original motivating idea was the seminal work of \cite{MR834612} where the differential Harnack inequality for the Schr\"odinger operator (and in particular for the heat equation) on a compact manifold with positive Ricci curvature was introduced. A sharp inequality was obtained, with equality attained precisely on the heat kernel, a self-similar solution of the heat equation. Motivated by the sharp inequality, Hamilton \cite{MR1316556} was lead to seek a sharp Harnack inequality with equality occurring precisely on solitons, the self-similar solutions of the mean curvature flow. The result is the definition of the Harnack quadratic:
\begin{equation}
\label{eq:harnack_quadratic}
Q = \partial_t \p - b^{ij} \nabla_i \p \nabla_j \p
\end{equation}
where \(b^{ij}\) is the inverse of the second fundamental form of a strictly convex hypersurface. Several authors, beginning with \cite{MR1362964} for the Ricci flow and \cite{MR1856176} for hypersurface flows, have investigated this quadratic \cite{2013arXiv1301.1543H,kotschwarharnack}, relating it to the second fundamental form of a degenerate metric on space-time though we will not pursue this line of investigation here.

The Li-Yau approach \cite{MR834612}, begins by considering the function \(u = \ln \p\). Following this approach in our situation, from the evolution of the speed \(\p\) given below in \cref{lem:evolution}, \cref{eq:delt_speed} we see that
\[
\partial_t u = \Box u + \p^{ij} \nabla_i u \nabla_j u + \p^{ij}(h^2)_{ij} + \p^{ij}g_{ij}
\]
where \(\Box\) is the elliptic operator \(\p^{ij} \nabla^2_{ij}\), \(\p^{ij}\) is the differential of the speed function (see the next section below), and \((h^2)_{ij}\) is a quadratic metric contraction of the second fundamental form. One now computes the evolution of \(\Box u - \p^{ij}(h^2)_{ij} - \p^{ij}g_{ij} = \partial_t u - \p^{ij} \nabla_i u \nabla_j u\) and seeks to apply the maximum principle to show that \(\partial_t u - \p^{ij} \nabla_i u \nabla_j u \geq - \tfrac{C}{t}\p\). It is (almost) immediately clear that this approach will not work since the evolution of \(u\) depends not only on the speed function \(\p\), but also on the \emph{time-dependent} metric and second fundamental form. Hence one needs to modify the Li-Yau quadratic \(\partial_t u - \p^{ij} \nabla_i u \nabla_j u\), introducing additional ``compensating terms'' for the maximum principle to work. To determine the appropriate terms, Hamilton supposed the equality case should hold for self-similar solutions and obtained \(Q\) as above which turns out to be very similar to the Li-Yau Harnack quadratic.

Taking a seemingly entirely different approach, working with respect to the Gauss parametrization used in \cite{MR1296393}, the Harnack quadratic is simply \(\partial_t \p\). Transforming back to the curvature flow parametrization (called the ``standard'' parametrization in \cite{MR1296393}), miraculously re-introduces \emph{exactly the missing term} and gives the Harnack inequality! This method is quite close in philosophy to the Li-Yau approach because everything is computed with respect to the time-independent standard metric on the sphere.

Now, our results take place on the sphere. The general procedure outlined above is to add terms to the Li-Yau Harnack quadratic in order to obtain a Harnack, and ideally one would like a sharp Harnack inequality with equality occurring precisely on solitons. Attempting to follow this procedure, one is lead immediately to the questions: 
\begin{enumerate}
\item What are the self-similar solutions on the sphere?
\item What sort of Gauss map parametrization, if any is available? 
\end{enumerate}

For the second question, note that the Gauss map may be considered as a map \(M_t \to \S^{n+1}\), the image of which is a time-dependent convex hypersurface (assuming that \(M_t\) is itself convex) \cite{Gerhardt:/2006}. Thus the Gauss map parametrization of \(M_t\) is not defined on fixed convex hypersurface, but rather a changing convex hypersurface of \(\S^{n+1}\) and this approach has significantly less utility than in the Euclidean case. There are various other possible candidates, but we found none that worked as well as in the Euclidean case except for certain special cases. It would be quite interesting to determine if there exists a ``canonical'' parametrization for strictly convex hypersurfaces of the sphere, compatible with the Harnack quadratic analogous to the Gauss map parametrization in Euclidean space.

The first question offers more possibilities. The usual symmetries considered in Euclidean space are isometries of the ambient space and space-time scalings. For the Harnack inequality, rotations are not so important, but translations and scalings are of utmost importance. On the sphere, if the initial area is smaller than the area of an equator, under suitable curvature assumptions, the hypersurface collapses to a point in finite time \cite[Theorem 0.1]{MR892052}, and thus cannot move by isometry. Otherwise, it converges smoothly to an equator and hence must have been an equator for all time (since it moves by isometries). Therefore, there are no closed, convex hypersurfaces of the sphere moving by isometry. This is the same in Euclidean space of course, and in that situation one may also consider translators. But these are non-compact hence do not exist in the sphere. 

As to scalings, these also do not exist on the sphere, but the family of shrinking geodesic spheres is ``sufficiently self-similar'' to be of interest. Here we are only interested in attaining a qualitative understanding of the connection between solitons and Harnack inequalities, so let us consider the relatively simpler case of the mean curvature flow. In this case, a family of shrinking geodesic spheres is given the graph \(\{\radialdistance = \radialdistance(t) = \arccos(e^{t})\}\) in geodesic polar coordinates around the equator (see e.g., \cite[equations (6.5), (6.11)]{Gerhardt:/2015} ). The mean curvature of such a family is
\[
H(t) = n\cot(\radialdistance(t)) = n\frac{e^{t}}{\sqrt{1 - e^{2t}}}.
\]
Then direct computation gives
\[
\del_t H - n H = -f(t) H
\]
where
\[
f(t) = \frac{n e^{2nt}}{1 - e^{2nt}} = \frac{1}{2t}(1 + 2nt + \cdots).
\]
On the other hand, the Harnack inequality (\cref{thm:harnack} with \(\p = H\), or \cite{IvakiBryan:08/2015}) states that
\[
\del_t H - n H \geq - \frac{1}{2t} H
\]
whenever \(\nabla H = 0\) as in the case of geodesic spheres.

Thus we find that the Harnack inequality gives in \cref{thm:harnack} is only ``sharp to first order''. The lack of a sharp Harnack is perhaps not too surprising given that the shrinking spheres are flowing by conformal transformation and not a true symmetry for our curvature flows.

Before moving on, let us make a few remarks on applications of Harnack inequalities. These have proven an important tool in a number of geometric evolution equations. For the Ricci flow on surfaces, it is an essential ingredient in proving that any initial metric on \(\S^2\) converges to a constant curvature metric under the normalized Ricci flow \cite{MR954419,MR1094458}. The Harnack inequality also plays an important role in Perelman's proof of Thurston's geometrization conjecture \cite{arXiv0211159}, with close connections to the entropy introduced there. See \cite{MR2251315} for a very nice account. This role is quite broad in that the Harnack inequality is an essential tool for understanding singularity formation of curvature flows in Euclidean space. Such singularities are modeled by ancient solutions, a particularly important class being solitons, which in turn are characterized by the equality case in the Harnack inequality \cite{MR1666878}. Importantly, apart from a flat factor, rescaling near a Type II singularity results in a convex, translating soliton and so the study of convex solutions is of great importance. Recently, similar results were obtained in the sphere \cite{MR3335392}, which combined with our work here could lead to greater understanding of singularity formation for curvature flows in the sphere, though the lack of a sharp Harnack is somewhat vexing.

In \cite{IvakiBryan:08/2015} we proved the classification result that any convex, ancient solution to the mean curvature flow on the sphere by using the Harnack inequality to obtain a curvature bound, and then an Aleksandrov reflection technique. The same classification result was obtained in \cite{MR3399098}, with a much shorter proof, applying the maximum principle to a pinching quantity. Previously, we remarked that the Harnack inequality/Aleksandrov reflection argument is quite general and should be broadly applicable. This is indeed the case, but in an upcoming paper \cite{BryanIvakiScheuer:0} we show that for all the flows for which we have obtained a Harnack inequality, there is, in fact, a very short classification proof similar to \cite{MR3399098}. Thus we defer consideration of ancient solutions to that paper, where a very general classification result using the rigidity result in \cite{MakowskiScheuer:/2013} and the Aleksandrov reflection technique from \cite{bryanlouie,IvakiBryan:08/2015} is proven. The result of that paper then yields classifications for \emph{any flows in the sphere for which a Harnack inequality holds}. At this stage, however, we do not know just how large a class of flows this is.

This paper is laid out as follows: in \cref{sec:prelim} we define our notational conventions and recall some standard definitions and identities. In \cref{sec:basic_evolution} give some standard evolution equations and commutators, and carry out the tedious task of computing the evolution of various quantities necessary for the main argument. The next section, \cref{sec:main_evolution} combines these computations into evolution equations for the Harnack quadratics we study. Then, applying these calculations we derive the Harnack inequalities in \cref{sec:Harnack}. In this section, we present several variants depending on the strength of our assumptions. To finish, we prove preservation of convexity in \cref{sec:convexity} for various flows in order to show that the assumption of convexity is a natural one, and there are many convex solutions of the flows considered here.

\section*{Acknowledgment}

The authors would like to thank Knut Smoczyk and the Institut f\"{u}r Differentialgeometrie at Leibniz Universität for hosting a research visit where part of this work took place. Bennet Chow was also very encouraging, suggesting that Harnack inequalities appear to be quite robust and should hold for a broad class of curvature flows, inspiring us to undertake the daunting calculations required. The first author would also like to thank the third author for the gift of a bottle of french wine, only to be opened upon completion of this paper which served as strong motivation for completion. The first author was a Riemann Fellow at the Riemann Center for Geometry and Physics, Leibniz Universit\"{a}t whilst this work was conducted. The work of the second author was supported by Austrian Science Fund (FWF) Project M1716-N25. 

\section{Preliminaries}
\label{sec:prelim}

Let $\overline{g}$ and $\overline{Rm}$ denote, respectively, the metric and the curvature tensor of $M_c$. Define \(M_t := x(M^n,t)\), and let \(g = x_t^{\ast} \overline{g}\) denote the induced metric on \(M\) with $\nabla$ the corresponding Levi-Civita connection. Write $\nu$ for the outer unit normal to $M_t$ and let \(\{\partial_i\}_{i=1}^n\) be a local coordinate frame on \(M\) which extends to a frame \(\{\partial_0 = \nu, (x_t)_{\ast} \partial_1, \cdots, (x_t)_{\ast} \partial_n\}\) on \(M_c\) in a neighborhood of \(M_t\).

Let Greek indices range from \(0\) to \(n\) and Latin indices range from \(1\) to \(n\). The Riemann curvature tensor of \(M_c\) satisfies \(\bar{R}_{\a\beta\gamma\theta} = c(\bar{g}_{\a\gamma}\bar{g}_{\beta\delta} - \bar{g}_{\a\theta}\bar{g}_{\beta\gamma})\). We may write the metric $g = \{g_{ij}\}$, second fundamental form $A = \{h_{ij}\}$, the Weingarten map $\mathcal{W} = h^i_j = g^{mi} h_{jm}$ and the Riemann curvature tensor $Rm_{ijkl}$ with respect to the given frame.

We shall write \(\nabla_i\) for covariant derivatives and also use the notation \(\nabla^i = g^{ik} \nabla_k\). Second covariant derivatives will be written \(\nabla^2_{ij} = \nabla_{\partial_i} \nabla_{\partial_j} - \nabla_{\nabla_{\partial_i} \partial_j}\) and \((\nabla^2)^i_j = g^{ik} \nabla^2_{kj}\).

The mean curvature of $M^n$ is the trace of the Weingarten map (equivalently the trace of the second fundamental form with respect to $g$), $H = g^{ij}h_{ij} = h^i_i$. We also use the following standard notation
\[
(h^2)_i^j = g^{mj}g^{rs}h_{ir}h_{sm},
\]
\[
(h^2)_{ij} = g_{kj} (h^2)_i^k = h^k_i h_{kj},
\]
\[
|A|^2 = g^{ij}g^{kl}h_{ik}h_{lj} = h_{ij}h^{ij}.
\]
Here, $\{g^{ij}\}$ is the inverse matrix of $\{g_{ij}\}.$ For a strictly convex hypersurface, \(A\) is strictly positive-definite and hence has a strictly positive-definite inverse, which we denote by
\[
b = \{b^{ij}\}.
\]
The relations between $A$, $Rm$, and $\overline{Rm}$ are given by the Gau{\ss} and Codazzi equations:
\[
\begin{split}
Rm_{ijkl} &= \overline{Rm}_{ijkl} + h_{ik}h_{jl} - h_{il}h_{jk} \\
&= c(\bar{g}_{ik}\bar{g}_{jl} - \bar{g}_{il}\bar{g}_{jk}) + h_{ik}h_{jl} - h_{il}h_{jk},
\end{split}
\]
\[
\nabla_i h_{jk} = \nabla_k h_{ij},
\]
valid for space forms. We also make use of the Ricci identity,
\[
{Rm^m}_{kij}  = \left(\nabla^2_{ij} \partial_k - \nabla^2_{ji} \partial_k\right)^m.
\]
We will need some notation for derivatives of the speed \(\varphi\). Let us write
\[
\varphi^{i}_{j} = \frac{\partial \varphi}{\partial h^{j}_{i}}
\]
for the first partial derivatives of \(\varphi\). We may also think of \(\varphi\) as a function of the metric and second fundamental form
\[
\varphi(g, h) = \varphi(g^{ik} h_{kj}).
\]
From this point of view, for the first and second partial derivatives, let us write
\[
\varphi^{ij} = \frac{\partial\varphi}{\partial h_{ij}}, \quad \varphi^{ij,kl} = \fr{\partial^2\varphi}{\partial h_{kl} \partial h_{ij}}.
\]
The trace of \(\varphi^{ij}\) with respect to the metric will be written
\[
\operatorname{tr}(\dot{\p}) = g_{ij} \varphi^{ij}.
\]
Let us also define the operator
$
\Box = \varphi^{ij} \nabla^2_{ij}.
$
This operator satisfies the product rule
\begin{equation}
\label{eq:productbox}
\Box (fg) = f \Box g + g \Box f + 2 \varphi^{ij} \nabla_i f \nabla_j g.
\end{equation}

We frequently make use, without comment, of the formula for differentiating an inverse
\[
\frac{\partial g^{ij}}{\partial g_{kl}} = - g^{kj} g^{il}.
\]
First derivatives of \(\varphi\) from the two perspectives are related by
\begin{equation}
\label{eq:delh}
\varphi^{ij} = \frac{\partial \varphi}{\partial h_l^k} \frac{\partial h_l^k}{\partial h_{ij}} = \varphi^l_k g^{ik} \delta^j_l = g^{ik} \varphi^j_k.
\end{equation}
and
\begin{equation}
\label{eq:delg}
\frac{\partial\varphi}{\partial g_{ij}} = \varphi^{l}_{k} \frac{\partial h^{k}_{l}}{\partial g_{ij}} = -\varphi^{l}_{k} g^{ki} g^{rj} h_{rl} = -\varphi^{li}h^{j}_{l}.
\end{equation}
We will also need the mixed second derivatives,
\begin{equation}
\label{eq:delhdelg}
\begin{split}
\frac{\partial \varphi^{ij}}{\partial g_{kl}} &= \frac{\partial}{\partial g_{kl}} \left(g^{sj} \varphi^{i}_{s} \right) = - g^{kj}g^{sl} \varphi^{i}_{s} - g^{sj} (\varphi^i_s)^{mk} h^l_m \\
&= - g^{kj} \varphi^{il} - g^{sj} g_{ns} \varphi^{in,mk} h^l_m \\
&= - g^{kj} \varphi^{il} - \varphi^{ij,mk} h^l_m,
\end{split}
\end{equation}
where we applied \cref{eq:delg} to \(\varphi^i_s\) in the first line.

Covariant derivatives of \(\varphi\) satisfy
\begin{equation}
\label{eq:delphi}
\nabla_k \varphi = \varphi^{ij} \nabla_k h_{ij}
\end{equation}
and the covariant derivative of the trace,
\begin{equation}
\label{eq:delPhi}
\nabla_k \operatorname{tr}(\dot{\p}) = g_{ij} \varphi^{ij,rs} \nabla_k h_{rs}.
\end{equation}

\section{Basic Evolution equations}
\label{sec:basic_evolution}

Following \cite{Andrews:09/1994, Chow:06/1991, Hamilton:/1995, Smoczyk:/1997}, in this section, we collect basic evolution equations that are needed to calculate the evolution of the following quantities
\[
\chi_1 =t(\partial_t \varphi- b^{ij} \nabla_i \varphi \nabla_j \varphi) +\delta\varphi
\]
and
\[
\chi_2 =t(\partial_t \varphi - b^{ij} \nabla_i \varphi \nabla_j \varphi - c \varphi \operatorname{tr}(\dot{\p})) +\delta\varphi
\]
where \(\delta \ne 0\) is an arbitrary, non-zero constant. The evolution equation of $\chi_1$ will be used for obtaining Harnack estimates for convex 1-homogeneous curvature speeds $\p$, and the evolution equation of $\chi_2$ will be employed for obtaining stronger Harnack estimates for flow by powers of the mean curvature $\p(H)=H^{p}$ with $p\in(0,1).$ Note that in Euclidean space $\chi_1=\chi_2.$

Let us make a few definitions to keep the calculations more manageable. Let
\[
\a_{ij} = \nabla^2_{ij} \varphi + \varphi(h^2)_{ij}, \quad \gamma_{ij} = b^{kl} \nabla_k \varphi \nabla_l h_{ij}, \quad \eta_{ij} = \a_{ij} - \gamma_{ij}
\]
and define
\[
\beta = \varphi^{ij} \a_{ij} = \Box\varphi + \varphi \varphi^{ij}(h^2)_{ij}, \quad \theta =  b^{ij} \nabla_i \varphi \nabla_j \varphi
\]
so that from the evolution of \(\varphi\) below (\cref{lem:evolution}, \cref{eq:delt_speed}) we may write our main Harnack quantities as
\[
\chi_1 = t(\partial_t\p- \theta) + \delta\varphi
\quad \mbox{and}\quad
\chi_2 = t(\beta - \theta) + \delta\varphi.
\]
We begin by recalling some standard evolution equations and commutators and then break the remaining calculation into several lemmas.

The evolution equations in the following lemma are standard and can be found in many places \cite{Andrews:09/1994, Chow:06/1991, Hamilton:/1995, Huisken:/1987a, Smoczyk:/1997}. The necessary tools are commuting derivatives, using the definition of the curvature tensor for space forms, the Gauss equation, and the Codazzi equation as described in the previous section. Compare also \cite[p.~94-95]{Gerhardt:/2006} and the formula \cite[eq.~(6.17)]{Gerhardt:01/1996}.
\begin{lemma}
\label{lem:evolution}
The following evolution equations hold
\begin{enumerate}
\item \label{eq:delt_metric} $\partial_tg_{ij} = -2\varphi h_{ij}$
\item \label{eq:delt_inversemetric} $\partial_t g^{ij} = 2\varphi h^{ij}$
\item \label{eq:delt_sff} $\partial_t h_{ij} = \nabla^2_{ij} \varphi - \varphi(h^2)_{ij} + c \varphi g_{ij}$
\item \label{eq:delt_weingarten} $\partial_t h_i^j = (\nabla^2)^j_i\varphi + \varphi(h^2)_i^j + c \varphi\delta_i^j = \a^j_i + c \varphi\delta_i^j$
\item \label{eq:delt_sff_box} \begin{align*}
\partial_t h_{ij} &= \Box h_{ij} + \varphi^{kl} (h^2)_{kl} h_{ij} - (\varphi^{kl}h_{kl} + \varphi) (h^2)_{ij} \\
& \quad + \varphi^{kl,rs}\nabla_i h_{kl}\nabla_j h_{rs} \\
& \quad + c \{(\varphi + \varphi^{kl}h_{kl}) g_{ij} - \operatorname{tr}(\dot{\p}) h_{ij}\}
\end{align*}
\item \label{eq:delt_weingarten_box} \begin{align*}
\partial_t h_i^j &= \Box h_i^j + \varphi^{kl} (h^2)_{kl} h_i^j - (\varphi^{kl}h_{kl} - \varphi) (h^2)_i^j \\
& \quad + \varphi^{kl,rs}\nabla_i h_{kl}\nabla^j h_{rs} \\
& \quad + c \{(\varphi + \varphi^{kl}h_{kl}) \delta_i^j - \operatorname{tr}(\dot{\p}) h_i^j\}
\end{align*}
\item \label{eq:delt_inversesff} \begin{align*}
\partial_t b^{ij} &= \Box b^{ij} - \varphi^{rs} (h^2)_{rs} b^{ij} + (\varphi^{kl}h_{kl} + \varphi) g^{ij} \\
& \quad - \left(2b^{lq}\varphi^{kp} + \varphi^{kl,pq}\right) b^{ir}b^{js} \nabla_r h_{kl} \nabla_s h_{pq} \\
& \quad - c \{(\varphi + \varphi^{kl}h_{kl}) b^{ir}b^{j}_{r} - \operatorname{tr}(\dot{\p}) b^{ij}\}
\end{align*}
\item \label{eq:delt_squaredsff} $\partial_t (h^2)_{ij} = h^k_j \nabla^2_{ik} \varphi + h^k_i \nabla^2_{jk} \varphi + h^k_j \varphi(h^2)_{ik} - h^k_i \varphi(h^2)_{jk} + 2c\varphi h_{ij}$
\item \label{eq:delt_speed} $\partial_t \varphi = \Box \varphi + \varphi\varphi^{ij}(h^2)_{ij} + c \varphi\varphi^{ij}g_{ij} = \beta + c\varphi\operatorname{tr}(\dot{\p}).$
\end{enumerate}
\end{lemma}

\begin{lemma}
\label{EvGamma}
The Christoffel symbols evolve according to
\begin{equation}
\partial_t {\G}^{k}_{ij} = -\varphi g^{kl} \nabla_l h_{ij} - g^{kl} h_{li} \nabla_j \varphi - g^{kl} h_{lj} \nabla_i \varphi + g^{kl} h_{ij} \nabla_l \varphi.
\end{equation}
\end{lemma}

\begin{proof}
In local coordinates, we have
\[
\G^{k}_{ij} = \frac{1}{2} g^{kl} \left(\partial_j g_{il} + \partial_i g_{jl} - \partial_l g_{ij}\right).
\]
Since $\partial_t \G^{k}_{ij}$ is a tensor, we may calculate using normal coordinates at any given point, at which \(\G^k_{ij} = 0\). Then we have
\[
\frac{1}{2} \partial_t g^{kl} (\partial_j g_{il} + \partial_i g_{jl} - \partial_l g_{ij}) = 2\varphi g^{kr} h_{rs} \G^s_{ij} = 0.
\]
from \cref{lem:evolution}, \cref{eq:delt_inversemetric}. Now commuting derivatives \([\partial_t, \partial_i] = 0\), and using the Codazzi equations we obtain
\[
\begin{split}
\partial_t {\G}^{k}_{ij} & = \frac{1}{2} g^{kl} \left(\partial_j \partial_t g_{il} + \partial_i \partial_t g_{jl} - \partial_l \partial_t g_{ij}\right) \\
&= - g^{kl} \left(\partial_j (\varphi h_{il}) + \partial_i (\varphi h_{jl}) - \partial_l (\varphi h_{ij})\right) \\
&= -\varphi g^{kl} \partial_l h_{ij} - g^{kl} h_{il} \partial_j \varphi  - g^{kl} h_{lj} \partial_i \varphi + g^{kl} h_{ij} \partial_l \varphi.
\end{split}
\]
The result follows since in normal coordinates, \(\nabla_i = \partial_i\) at our given point. 
\end{proof}

We require the commutators \([\nabla, \Box]\) and \([\partial_t, \Box]\). Without further comment we will also use the fact that \([\partial_t, \nabla] f = 0\) for any smooth function \(f\).

\begin{lemma}
\label{lem:gradBox}
For every smooth function $f$, the commutation relation
\[
\begin{split}
([\nabla, \Box]f)_i &= \nabla_i \Box f - (\Box \nabla f)_i\\ 
                &= \varphi^{kl,rs} \nabla_i h_{rs} \nabla^2_{kl} f + \varphi^{kl}\left(h^{m}_{l}h_{ki} - h_{kl}h^{m}_{i}\right) \nabla_m f \\
&\quad + c \varphi^{kl} g_{ki} \nabla_l f - c \operatorname{tr}(\dot{\p}) \nabla_i f 
\end{split}
\]
holds, where \(\nabla f = df\) is the covariant derivative of \(f\) and the subscript \(i\) refers to the \(i\)'th component of a one-form.
\end{lemma}
\begin{proof}
From the Ricci identities
\[
{Rm^m}_{kij}  = \left(\nabla^2_{i, j} \partial_k - \nabla^2_{ji} \partial_k\right)^m
\]
we obtain that the $3$-tensor $\nabla^3_{kli}f-\nabla^3_{kil}f$
is given by
\[
\nabla^3_{kli}f-\nabla^3_{kil}f={Rm^m}_{kli}\nabla_m f
\]
and thus we obtain
\[
\nabla_i (\varphi^{kl} \nabla^2_{kl} f) - (\varphi^{kl}(\nabla^2_{kl} \nabla f))_i = \varphi^{kl,rs} \nabla_i h_{rs} \nabla^2_{kl}f + \varphi^{kl}{Rm^{m}}_{kli} \nabla_m f.
\]
From the Gauss equation we obtain
\[
\begin{split}
{Rm^{m}}_{kli} \nabla_m f &= \left(c\left(g^{pm}g_{ki}g_{pl}  - g^{pm}g_{pi}g_{kl}\right) + g^{pm} h_{pl}h_{ki} - g^{pm}h_{kl}h_{pi}\right) \nabla_m f \\
&= c\left(g_{ki} \nabla_l f - g_{kl} \nabla_i f\right) + \left(h^{m}_{l}h_{ki} - h_{kl}h^{m}_{i}\right) \nabla_m f.
\end{split}
\]
\end{proof}
\begin{lemma}
\label{lem:deltBox}
The following commutation relation holds
\[
\begin{split}
[\partial_t, \Box] \varphi &= (\partial_{t}\Box - \Box\partial_{t}) \varphi = \varphi^{ij,kl} \nabla^2_{ij} \varphi (\a_{kl} + c \varphi g_{kl}) \\
&\quad + 2\varphi^{ij}h^{k}_{i} (\varphi \nabla^2_{k
j} \varphi + \nabla_k \varphi \nabla_j \varphi) + (\varphi - \varphi^{ij}h_{ij})| \nabla\varphi|^{2}.
\end{split}
\]
\end{lemma}
\begin{proof}
First, let us calculate the evolution of \(\varphi^{ij}\), which will also prove useful later. From the mixed derivative \cref{eq:delhdelg}, the evolution of the metric (\cref{lem:evolution}, \cref{eq:delt_metric}), and the evolution of the second fundamental form (\cref{lem:evolution}, \cref{eq:delt_sff}) we compute
\begin{equation}
\label{eq:deltBox}
\begin{split}
\partial_{t} \varphi^{ij} &= \varphi^{ij,kl} \partial_t h_{kl} + \frac{\partial\varphi^{ij}}{\partial g_{kl}} \partial_t g_{kl} \\
&= \varphi^{ij,kl} \left(\nabla^2_{kl} \varphi - \varphi(h^2)_{kl} + c \varphi g_{kl}\right) + 2\varphi \varphi^{ij,kl} h_{lm}h^{m}_{k} + 2\varphi\varphi^{jk}g^{li}h_{kl} \\
&= \varphi^{ij,kl} \left(\nabla^2_{kl} \varphi + \varphi(h^2)_{kl} + c \varphi g_{kl}\right) + 2\varphi\varphi^{jk}h^{i}_{k} \\
&= \varphi^{ij,kl} \left(\a_{kl} + c \varphi g_{kl}\right) + 2\varphi\varphi^{jk}h^{i}_{k}.
\end{split}
\end{equation}
Next, the commutator of \(\partial_t\) and \(\nabla^2_{ij}\) is given by,
\begin{equation}
\label{eq:deltnabla2}
\begin{split}
\left(\partial_{t}\nabla^2_{ij} - \nabla^2_{ij}\partial_{t}\right) \varphi &= \partial_t \left(\nabla_i \nabla_j \varphi - \nabla_{\nabla_i \partial_j} \varphi\right) - \nabla_i \nabla_j \partial_t \varphi + \nabla_{\nabla_i \partial_j} \partial_t \varphi \\
&= - \partial_t \left(\G_{ij}^k \nabla_{\partial_k} \varphi\right) + \G_{ij}^k \nabla_{\partial_k} \partial_t \varphi \\
&= - \nabla_k \varphi \partial_t \G_{ij}^k.
\end{split}
\end{equation}
We obtain from \cref{eq:deltBox}, \cref{eq:deltnabla2}, and the evolution of the Christoffel symbols \cref{EvGamma},
\[
\begin{split}
\left(\partial_{t}\Box - \Box\partial_{t}\right) \varphi &= \left(\partial_{t}\varphi^{ij}\right) \nabla^2_{ij} \varphi - \varphi^{ij}\nabla_k \varphi\partial_t \G^{k}_{ij} \\
&= \left[\varphi^{ij,kl} \left(\a_{kl} + c \varphi g_{kl}\right) + 2\varphi\varphi^{jk}h^{i}_{k}\right] \nabla^2_{ij} \varphi \\
&\quad + \varphi^{ij} \nabla_k \varphi \left(\varphi g^{kl} \nabla_l h_{ij} + h^k_i \nabla_j \varphi + h^k_j \nabla_i \varphi - g^{kl} h_{ij} \nabla_l \varphi\right) \\
&= \varphi^{ij,kl} \nabla^2_{ij} \varphi \left(\a_{kl} + c \varphi g_{kl}\right) \\
&\quad + 2\varphi^{jk}h^{i}_{k}\varphi \nabla^2_{ij} \varphi + h^k_i \varphi^{ij} \nabla_k \varphi \nabla_j \varphi + h^k_j \varphi^{ij} \nabla_k \varphi \nabla_i \varphi \\
&\quad + \varphi g^{kl}\nabla_k \varphi \varphi^{ij} \nabla_l h_{ij} - \varphi^{ij} h_{ij} g^{kl}\nabla_k \varphi \nabla_l \varphi.
\end{split}
\]
The result now follows from \(|\nabla \varphi|^2 = g^{kl}\nabla_k \varphi \nabla_l \varphi\) and \(\nabla_l \varphi = \varphi^{ij} \nabla_l h_{ij}\) by the chain rule.
\end{proof}
The next ingredient is the evolution of the covariant derivative, \(\nabla \varphi = d\varphi\).
\begin{lemma}
\label{lem:Evgradphi}
There holds
\[
\begin{split}
\left((\partial_{t}-\Box)\nabla\varphi\right)_{i} &= \varphi^{kl,rs}\nabla_i h_{rs} \a_{kl} + 2 \varphi^{kl} b^{rs} \varphi(h^2)_{rl} \nabla_i h_{ks} \\
&\quad + \varphi^{kl}(h^2)_{kl}\nabla_i \varphi + \left(\varphi^{kl}h^{m}_{l}h_{ki} - \varphi^{kl}h_{kl}h^{m}_{i}\right) \nabla_m \varphi\\
&\quad + c\left(\varphi^{kl}g_{ki} \nabla_l \varphi + \varphi \nabla_i \operatorname{tr}(\dot{\p})\right).
\end{split}
\]
\end{lemma}
\begin{proof}
Using the evolution of \(\varphi\) \cref{lem:evolution}, \cref{eq:delt_speed}, the derivative of \(\operatorname{tr}(\dot{\p})\) \cref{eq:delPhi}, and the commutator \([\nabla, \Box]\) from \cref{lem:gradBox}, we compute
\[
\begin{split}
\partial_{t}\nabla_i \varphi - (\Box\nabla \varphi)_{i} &= \nabla_i \partial_t \varphi - \nabla_i \Box \varphi + ([\nabla, \Box] \varphi)_i \\
&= \nabla_i \left(\Box\varphi + \varphi^{kl}(h^2)_{kl}\varphi + c \operatorname{tr}(\dot{\p})\varphi\right) - \nabla_i (\Box\varphi) \\
&\quad + \varphi^{kl,rs} \nabla_i h_{rs} \nabla^2_{kl} \varphi + c\varphi^{kl}g_{ki} \nabla_l \varphi - c\operatorname{tr}(\dot{\p})\nabla_i \varphi \\
&\quad + (\varphi^{kl}h^{m}_{l}h_{ki} - \varphi^{kl}h_{kl}h^{m}_{i}) \nabla_m \varphi \\
&= \varphi^{kl}(h^2)_{kl}\nabla_i \varphi + \varphi^{kl,rs}\nabla_i h_{rs} (h^2)_{kl}\varphi + \varphi\varphi^{kl}(h^s_l \nabla_i h_{ks} + h^r_k \nabla_i h_{rl})\\&\quad  + c \varphi\nabla_i\operatorname{tr}(\dot{\p})
+ \varphi^{kl,rs} \nabla_i h_{rs} \nabla^2_{kl} \varphi + c\varphi^{kl}g_{ki}\nabla_l \varphi \\
&\quad + (\varphi^{kl}h^{m}_{l}h_{ki} - \varphi^{kl}h_{kl}h^{m}_{i}) \nabla_m \varphi \\
&= \varphi^{kl,rs}\nabla_i h_{rs} \left((h^2)_{kl}\varphi + \nabla^2_{kl} \varphi\right) + 2 \varphi\varphi^{kl} h^s_l \nabla_i h_{ks} \\
&\quad + \varphi^{kl}(h^2)_{kl}\nabla_i \varphi + (\varphi^{kl}h^{m}_{l}h_{ki} - \varphi^{kl}h_{kl}h^{m}_{i}) \nabla_m \varphi \\
&\quad + c \left(\varphi\nabla_i\operatorname{tr}(\dot{\p}) + \varphi^{kl}g_{ki}\nabla_l \varphi\right),
\end{split}
\]
where in the third equality, we used
$
\nabla_i (h^2)_{kl} = \nabla_i (g^{sr} h_{ks} h_{rl}) = h^s_l \nabla_i h_{ks} + h^r_k \nabla_i h_{rl}
$
and in the last equality we used
$
b^{rs} (h^2)_{rl} = b^{rs} h_{rm} h^m_l = \delta^s_m h^m_l = h^s_l.
$
\end{proof}
Now we may proceed to the calculations of \(\partial_t \beta\) and \(\partial_t \theta\).
\begin{lemma}
\label{lem:evbeta}
The quantity $\beta$
satisfies
\[
\begin{split}
(\partial_{t} - \Box)\beta &= \left(\varphi^{ij}(h^2)_{ij} + c\operatorname{tr}(\dot{\p}) \right)\beta \\
&\quad + (\varphi - \varphi^{ij}h_{ij}) |\nabla\varphi|^{2} + 2\varphi^{ij}h^{k}_{i}\nabla_k \varphi \nabla_j \varphi \\
&\quad + \varphi^{ij,kl} \a_{ij} \a_{kl} \\
&\quad + 2b^{il}\varphi^{jk} (2\nabla^2_{ij}\varphi\varphi(h^2)_{kl} + \varphi(h^2)_{ij}\varphi(h^2)_{kl}) \\
&\quad + cR_{\beta},
\end{split}
\]
where
$
R_{\beta} = \varphi \Box \operatorname{tr}(\dot{\p}) + 2\varphi^{kl} \nabla_k \operatorname{tr}(\dot{\p}) \nabla_l \varphi + \varphi \varphi^{ij,kl}g_{kl} \a_{ij} + 2\varphi^{2}\varphi^{ij}h_{ij}.
$
\end{lemma}
\begin{proof}
Let us break up the calculation of
\[
(\partial_{t} - \Box)\beta =  \partial_{t}\Box\varphi + \partial_{t} (\varphi\varphi^{ij} (h^2)_{ij}) - \Box\Box\varphi - \Box(\varphi\varphi^{ij} (h^2)_{ij})
\]
into smaller pieces. First, we have lots of nice cancellation. Using the evolution of \(\varphi\) from \cref{lem:evolution}, \cref{eq:delt_speed} and the commutator relation from \cref{lem:deltBox} we have,
\begin{equation}
\label{eq:deltbeta1}
\begin{split}
&\partial_{t}\Box\varphi - \Box\Box\varphi - \Box(\varphi\varphi^{ij} (h^2)_{ij})\\ =\ &\Box\partial_t\varphi - \Box\Box\varphi - \Box(\varphi\varphi^{ij} (h^2)_{ij}) + [\partial_t, \Box] \varphi \\
=\ &\Box(\Box \varphi + \varphi\varphi^{ij}(h^2)_{ij} + c \varphi\operatorname{tr}(\dot{\p})) - \Box\Box\varphi - \Box(\varphi\varphi^{ij} (h^2)_{ij})\\
\quad &+ \varphi^{ij,kl} \nabla^2_{ij} \varphi (\a_{kl} + c \varphi g_{kl}) + 2\varphi^{ij}h^{k}_{i} (\varphi \nabla^2_{kj} \varphi + \nabla_k \varphi \nabla_j \varphi)\\ 
\quad &+ (\varphi - \varphi^{ij}h_{ij})| \nabla\varphi|^{2} \\
=\ &c (\operatorname{tr}(\dot{\p}) \Box \varphi + \varphi \Box \operatorname{tr}(\dot{\p}) + 2 \varphi^{kl} \nabla_k \operatorname{tr}(\dot{\p}) \nabla_l \varphi) \\
\quad &+ \varphi^{ij,kl} \nabla^2_{ij} \varphi (\a_{kl} + c \varphi g_{kl}) + 2\varphi^{ij}b^{kl} \varphi (h^2)_{il} \nabla^2_{kj} \varphi\\
\quad  &+ 2\varphi^{ij}h^{k}_{i} \nabla_k \varphi \nabla_j \varphi + (\varphi - \varphi^{ij}h_{ij})| \nabla\varphi|^{2} \\
=\ &c \operatorname{tr}(\dot{\p}) \Box \varphi + (\varphi - \varphi^{ij}h_{ij})| \nabla\varphi|^{2}  + \varphi^{ij,kl} \nabla^2_{ij} \varphi \a_{kl} \\
\quad &+ 2\varphi^{ij}b^{kl} \varphi (h^2)_{il} \nabla^2_{kj} \varphi + 2\varphi^{ij}h^{k}_{i} \nabla_k \varphi \nabla_j \varphi \\
\quad &+ c(\varphi \Box \operatorname{tr}(\dot{\p}) + 2 \varphi^{kl} \nabla_k \operatorname{tr}(\dot{\p}) \nabla_l \varphi + \varphi \varphi^{ij,kl} g_{kl} \nabla^2_{ij} \varphi)
\end{split}
\end{equation}
using, in the third equality, the product rule for \(\Box\) \cref{eq:productbox} and \(h^k_i = h^m_i b^{kl}h_{ml} = b^{kl} (h^2)_{il}\) since \(b\) is the inverse of \(A\).

Next from \cref{eq:deltBox} and \cref{lem:evolution}, \cref{eq:delt_squaredsff} we obtain
\[
\begin{split}
\partial_{t} (\varphi^{ij}(h^2)_{ij}) &= \partial_{t}(\varphi^{ij}) (h^2)_{ij} + \varphi^{ij} \partial_t (h^2)_{ij} \\
&= \left(\varphi^{ij,kl} \left(\a_{kl} + c \varphi g_{kl}\right) + 2\varphi\varphi^{jk}h^{i}_{k}\right) (h^2)_{ij} \\
&\quad + \varphi^{ij} \left(h^k_j \nabla^2_{i
k} \varphi + h^k_i \nabla^2_{jk} \varphi + h^k_j \varphi(h^2)_{ik} - h^k_i \varphi(h^2)_{jk} + 2c\varphi h_{ij}\right) \\
&= \varphi^{ij,kl} (h^2)_{ij} \left(\a_{kl} + c \varphi g_{kl}\right) + 2\varphi\varphi^{jk} b^{il} (h^2)_{lk} (h^2)_{ij} \\
&\quad + 2 \varphi^{ij} b^{kl} (h^2)_{il} \nabla^2_{jk} \varphi  + 2c\varphi\varphi^{ij}h_{ij}
\end{split}
\]
again using \(h^k_i = b^{kl} (h^2)_{il}\) in the last equality.

The remaining term we need to compute is thus
\begin{equation}
\label{eq:deltbeta2}
\begin{split}
\partial_{t} (\varphi \varphi^{ij}(h^2)_{ij}) &= (\partial_{t} \varphi) \varphi^{ij}(h^2)_{ij} + \varphi \partial_t (\varphi^{ij} (h^2)^{ij}) \\
&= (\beta + c \varphi\operatorname{tr}(\dot{\p})) \varphi^{ij}(h^2)_{ij} \\
&\quad + \varphi \left[\varphi^{ij,kl} (h^2)_{ij} \left(\a_{kl} + c \varphi g_{kl}\right) + 2\varphi\varphi^{jk} b^{il} (h^2)_{lk} (h^2)_{ij} \right.\\
&\quad \left. + 2 \varphi^{ij} b^{kl} (h^2)_{il} \nabla^2_{jk} \varphi  + 2c\varphi\varphi^{ij}h_{ij}\right]. \\
&= (\beta + c \varphi\operatorname{tr}(\dot{\p})) \varphi^{ij}(h^2)_{ij} + \varphi^{ij,kl} \varphi (h^2)_{ij} \a_{kl}  \\
&\quad + 2 \varphi\varphi^{ij} b^{kl} (h^2)_{il} \nabla^2_{jk} \varphi + 2\varphi^{jk} b^{il} \varphi (h^2)_{lk} \varphi (h^2)_{ij} \\
&\quad + c \left[\varphi \varphi^{ij,kl} g_{kl} \varphi (h^2)_{ij} + 2c\varphi^2\varphi^{ij}h_{ij}\right].
\end{split}
\end{equation}

Now we add \cref{eq:deltbeta1} and \cref{eq:deltbeta2} together line by line to complete the proof.
\end{proof}

\begin{lemma}
\label{lem:Evtheta}
The quantity $\theta$ evolves according to
\[
\begin{split}
(\partial_{t} - \Box)\theta &= (\varphi^{ij}(h^2)_{ij} + c\operatorname{tr}(\dot{\p}))\theta \\
&\quad + (\varphi - \varphi^{ij}h_{ij})|\nabla\varphi|^{2} + 2\varphi^{ij}h^{k}_{i}\nabla_k\varphi\nabla_j\varphi \\
&\quad - \varphi^{kl,ij} (\gamma_{ij}\gamma_{kl}  - 2\a_{ij} \gamma_{kl}) - 2b^{il} \varphi^{jk} \left(\gamma_{ij} \gamma_{kl} - 2\a_{ij} \gamma_{kl} + \nabla^2_{ij}\varphi\nabla^2_{kl}\varphi\right) \\
&\quad + cR_{\theta},
\end{split}
\]
\end{lemma}
where
$
R_{\theta} = -(\varphi^{kl}h_{kl} + \varphi)b^{ir}b^{j}_{r}\nabla_i \varphi\nabla_j\varphi + 2 b^{j}_{k}\varphi^{kl}\nabla_l\varphi\nabla_j\varphi + 2 \varphi\varphi^{ij,kl} g_{ij} \gamma_{kl}.
$
\begin{proof}
Again using the product rule for \(\Box\), \cref{eq:productbox} and the symmetry \(b^{ij} = b^{ji}\), we have
\begin{equation}
\label{eq:delt_theta}
\begin{split}
(\partial_{t} - \Box)\theta &= (\partial_{t}b^{ij} - \Box b^{ij})\nabla_i \varphi\nabla_j\varphi + b^{ij} (\partial_{t} - \Box) (\nabla\varphi \otimes \nabla\varphi)_{ij} \\
&\quad - 2 \varphi^{kl} \nabla_k b^{ij} \nabla_l (\nabla \varphi \otimes \nabla\varphi)_{ij} \\
&= (\partial_{t}b^{ij} - \Box b^{ij})\nabla_i \varphi\nabla_j\varphi + 2 b^{ij} (\partial_{t} - \Box) (\nabla\varphi)_i \nabla_j\varphi\\ 
&\quad - 2 b^{ij} \varphi^{kl} \nabla^2_{ik} \varphi \nabla^2_{lj} \varphi - 4 \varphi^{kl} \nabla_k b^{ij} \nabla^2_{il} \varphi \nabla_j\varphi \\
&= (\partial_{t}b^{ij} - \Box b^{ij})\nabla_i \varphi\nabla_j\varphi + 2 b^{ij} (\partial_{t} - \Box) (\nabla\varphi)_i \nabla_j\varphi \\
&\quad - 2 b^{ij} \varphi^{kl} \nabla^2_{ik} \varphi \nabla^2_{lj} \varphi + 4 \varphi^{kl} b^{ip}b^{jq} \nabla_k h_{pq} \nabla^2_{il} \varphi \nabla_j\varphi \\
&= (\partial_{t}b^{ij} - \Box b^{ij})\nabla_i \varphi\nabla_j\varphi + 2 b^{ij} (\partial_{t} - \Box) (\nabla\varphi)_i \nabla_j\varphi \\
&\quad - 2 b^{ij} \varphi^{kl} \nabla^2_{ik} \varphi \nabla^2_{lj} \varphi + 4 \varphi^{kl} b^{ip}\gamma_{pk} \nabla^2_{il} \varphi,
\end{split}
\end{equation}
where in the second to last equality we used the formula for the derivative of the inverse \(b^{ij}\) of \(h_{ij}\) and the Codazzi equation in the last line, producing \(b^{jq} \nabla_k h_{pq} \nabla_j \varphi = b^{jq} \nabla_q h_{pk} \nabla_j \varphi = \gamma_{pk}\). The first term in final line appears on the second to last line of the statement of the lemma (with indices relabelled). The second term is part of \(4 b^{il}\varphi^{jk} \a_{ij} \gamma_{kl}\) in the second to last line. So we must deal with the first two terms and show they add to the remainder of the statement. For the first term, we use the evolution of \(b^{ij}\) from \cref{lem:evolution}, \cref{eq:delt_inversesff} to calculate
\begin{equation}
\label{eq:delt_theta1}
\begin{split}
(\partial_{t}b^{ij} - \Box b^{ij})\nabla_i \varphi\nabla_j\varphi &= \nabla_i \varphi \nabla_j \varphi\left[-\varphi^{rs} (h^2)_{rs} b^{ij} + (\varphi^{kl}h_{kl} + \varphi) g^{ij} \right. \\
& \quad - \left(2b^{lq}\varphi^{kp} + \varphi^{kl,pq}\right) b^{ir}b^{js} \nabla_r h_{kl} \nabla_s h_{pq} \\
& \quad - \left. c \{(\varphi + \varphi^{kl}h_{kl}) b^{ir}b^{j}_{r} - \operatorname{tr}(\dot{\p}) b^{ij}\}\right] \\
&= \left(c\operatorname{tr}(\dot{\p}) - \varphi^{rs}(h)^2_{rs}\right) \theta\\ & \quad - (2b^{lq}\varphi^{kp} + \varphi^{kl,pq}) b^{ir}b^{js}\nabla_i\varphi\nabla_j\varphi\nabla_rh_{kl}\nabla_s h_{pq} \\
&\quad + (\varphi^{kl}h_{kl} + \varphi)|\nabla\varphi|^{2} - c(\varphi^{kl}h_{kl} + \varphi)b^{ir}b^{j}_{r}\nabla_i \varphi\nabla_j\varphi \\
&= \left(c\operatorname{tr}(\dot{\p}) - \varphi^{rs}(h)^2_{rs}\right) \theta  + (\varphi^{kl}h_{kl} + \varphi)|\nabla\varphi|^{2} \\
&\quad - \varphi^{kl,pq} \gamma_{kl} \gamma_{pq} - 2b^{lq}\varphi^{kp} \gamma_{kl} \gamma_{pq}  \\ 
&\quad -c(\varphi^{kl}h_{kl} + \varphi)b^{ir}b^{j}_{r}\nabla_i \varphi\nabla_j\varphi.
\end{split}
\end{equation}
For the second term, from the evolution of \(\nabla\varphi\) in \cref{lem:Evgradphi}, we have
\begin{equation}
\label{eq:delt_theta2}
\begin{split}
&2 b^{ij} (\partial_{t} - \Box) (\nabla\varphi)_i \nabla_j\varphi\\  
=\ &2 b^{ij} \nabla_j\varphi \big[\varphi^{kl,rs}\nabla_i h_{rs} \a_{kl}\varphi + 2 \varphi^{kl} b^{rs} \varphi(h^2)_{rl} \nabla_i h_{ks}  \\
\quad &+ \varphi^{kl}(h^2)_{kl}\nabla_i \varphi + \left(\varphi^{kl}h^{m}_{l}h_{ki} - \varphi^{kl}h_{kl}h^{m}_{i}\right) \nabla_m \varphi\\
\quad &+ c\left(\varphi^{kl}g_{ki} \nabla_l \varphi + \varphi \nabla_i \operatorname{tr}(\dot{\p})\right)\big] \\
=\ &2 b^{ij} \nabla_j\varphi \varphi^{kl}(h^2)_{kl}\nabla_i \varphi \\
\quad &- 2 b^{ij} \nabla_j\varphi \varphi^{kl}h_{kl}h^{m}_{i} \nabla_m \varphi + 2 b^{ij} \nabla_j\varphi \varphi^{kl}h^{m}_{l}h_{ki} \nabla_m \varphi \\
\quad &+ 2 b^{ij} \nabla_j\varphi \varphi^{kl,rs}\nabla_i h_{rs} \a_{kl}  + 4 b^{ij} \nabla_j\varphi \varphi^{kl} b^{rs} \varphi(h^2)_{rl} \nabla_i h_{ks} \\
\quad &+ c\left(2 b^{ij} \nabla_j\varphi \varphi^{kl}g_{ki} \nabla_l \varphi + 2 b^{ij} \nabla_j\varphi \varphi \nabla_i \operatorname{tr}(\dot{\p})\right) \\
=\ &2 \varphi^{kl}(h^2)_{kl}\theta - 2 \varphi^{kl}h_{kl} |\nabla\varphi|^2 + 2 \varphi^{kl} h^{m}_{l} \nabla_k\varphi \nabla_m \varphi \\
\quad &+ 2 \varphi^{kl,rs} \gamma_{rs} \a_{kl} + 4 b^{rs} \varphi^{kl} \gamma_{ks} \varphi(h^2)_{rl} \\
\quad &+ c\left(2 \varphi^{kl} b^j_k \nabla_j\varphi \nabla_l \varphi + 2 \varphi b^{ij} \nabla_j\varphi \nabla_i \operatorname{tr}(\dot{\p})\right)
\end{split}
\end{equation}
using the definitions of \(\theta, \a_{ij}\) and \(\gamma_{ij}\) as well as \(b^{ij}h_{ki} \nabla_j \varphi = \delta^j_k \nabla_j \varphi = \nabla_k \varphi\), and \(b^{ij} h^m_i = b^{ij} g^{mp}h_{pi} = \delta^j_p g^{mp} = g^{mj}\) in the last equality.

The proof is now completed by adding \cref{eq:delt_theta1} and \cref{eq:delt_theta2} line by line and adding also the final line from \cref{eq:delt_theta}.
\end{proof}

\section{Main evolution equations}
\label{sec:main_evolution}

We start this section by calculating the evolution equations of $\chi_2$ and its slight modification, $\chi_3$, which will be employed to obtain Harnack estimates for flow by powers of the mean curvature. We will then focus on the evolution equation of $\chi_1$ which will enable us to deduce (weak) Harnack estimates for all 1-homogeneous convex speeds.
\begin{proposition}
\label{thm:Evchi}
Let $\delta \neq 0.$ The quantity
$
\chi_2 = t(\beta - \theta) + \delta\varphi
$
satisfies
\eq{\label{thm:Evchi1}
\partial_t \chi_2 -\Box\chi_2 &= \left(\frac{\b-\t}{\delta\varphi} +\varphi^{ij}(h^2)_{ij} + c\operatorname{tr}(\dot{\p})\right)\chi_2 \\
& \quad + t\left(\varphi^{ij,kl} + 2b^{il}\varphi^{jk} - \frac{\varphi^{ij}\varphi^{kl}}{\delta\varphi}\right)\eta_{ij}\eta_{kl} + tc R,
}
where
\[
\eta_{ij} = \a_{ij} - \gamma_{ij} = \nabla^2_{ij}\varphi + (h^2)_{ij}\varphi - b^{rs}\nabla_r h_{ij}\nabla_s \varphi
\]
and
\[
\begin{split}
R &= R_{\beta} - R_{\theta} \\
&= \varphi \Box \operatorname{tr}(\dot{\p}) + 2\varphi^{kl} \nabla_k \operatorname{tr}(\dot{\p}) \nabla_l \varphi + \varphi \varphi^{ij,kl}g_{kl} (\nabla^2_{ij} \varphi + \varphi(h^2)_{ij})\\ 
&\quad- 2\varphi \varphi^{ij,kl}g_{kl} b^{rs}\nabla_r h_{ij}\nabla_s \varphi + 2\varphi^{2}\varphi^{ij}h_{ij}\\
&\quad  +(\varphi^{kl}h_{kl} + \varphi)b^{ir}b^{j}_{r}\nabla_i \varphi\nabla_j\varphi - 2 b^{j}_{k}\varphi^{kl}\nabla_l\varphi\nabla_j\varphi.
\end{split}
\]
\end{proposition}
\begin{proof}
We have
$
(\partial_t - \Box)\chi_2 = \beta - \theta + t(\partial_{t} - \Box)(\beta - \theta) + \delta(\partial_t \varphi - \Box\varphi).
$
First of all, the evolution equation for \(\varphi\), \cref{lem:evolution}, \cref{eq:delt_speed} gives us
\[
\delta(\partial_t \varphi - \Box\varphi) = \left(\varphi^{ij}(h^2)_{ij} + c\operatorname{tr}(\dot{\p})\right)\delta\varphi.
\]
Next, we note that
$
\varphi^{ij} \eta_{ij} = \beta - \theta
$
since \(\nabla_r \varphi = \varphi^{ij} \nabla_r h_{ij}\). Putting the two equations above together gives
\begin{equation}
\label{eq:deltchi1}
\begin{split}
&\beta - \theta + \delta(\partial_t \varphi - \Box\varphi) \\
=\ &\left(\frac{\beta-\theta}{\delta\varphi} + \varphi^{ij}(h^2)_{ij} + c\operatorname{tr}(\dot{\p})\right)\delta\varphi + t \frac{(\beta - \theta)^2}{\delta\varphi} - t \frac{(\varphi^{ij}\eta_{ij})^2}{\delta\varphi} \\
=\ &\frac{\beta-\theta}{\delta\varphi} \chi_2 + \left(\varphi^{ij}(h^2)_{ij} + c\operatorname{tr}(\dot{\p})\right)\delta\varphi  - t \frac{\varphi^{ij}\varphi^{kl}}{\delta\varphi} \eta_{ij}\eta_{kl}.
\end{split}
\end{equation}
The remaining term \(t(\partial_{t} - \Box)(\beta - \theta)\) is now just bookkeeping. Recall, \cref{lem:evbeta} states that
\begin{align*}
(\partial_{t} - \Box)\beta &= \left(\varphi^{ij}(h^2)_{ij} + c\operatorname{tr}(\dot{\p}) \right)\beta  & (A) \\
&\quad + (\varphi - \varphi^{ij}h_{ij}) |\nabla\varphi|^{2} + 2\varphi^{ij}h^{k}_{i}\nabla_k \varphi \nabla_j \varphi  & (B) \\
&\quad + \varphi^{ij,kl} \a_{ij} \a_{kl} & (C) \\
&\quad + 2b^{il}\varphi^{jk} (2\nabla^2_{ij}\varphi\varphi(h^2)_{kl} + \varphi(h^2)_{ij}\varphi(h^2)_{kl}) & (D) \\
&\quad + cR_{\beta}  & (E) \\
\intertext{while \cref{lem:Evtheta} states that}
(\partial_{t} - \Box)\theta &= (\varphi^{ij}(h^2)_{ij} + c\operatorname{tr}(\dot{\p}))\theta & (A') \\
&\quad + (\varphi - \varphi^{ij}h_{ij})|\nabla\varphi|^{2} + 2\varphi^{ij}h^{k}_{i}\nabla_k\varphi\nabla_j\varphi & (B') \\
&\quad - \varphi^{kl,ij} (\gamma_{ij}\gamma_{kl}  - 2\a_{ij} \gamma_{kl}) & (C') \\
&\quad - 2b^{il} \varphi^{jk} \left(\gamma_{ij} \gamma_{kl} - 2\a_{ij} \gamma_{kl} + \nabla^2_{ij}\varphi\nabla^2_{kl}\varphi\right) & (D') \\
&\quad + cR_{\theta}. & (E')
\end{align*}
Subtracting line by line, we have
\begin{align*}
(A) - (A') &= \left(\varphi^{ij}(h^2)_{ij} + c\operatorname{tr}(\dot{\p}) \right)\beta - \left(\varphi^{ij}(h^2)_{ij} + c\operatorname{tr}(\dot{\p}) \right)\theta \\
        &= \left(\varphi^{ij}(h^2)_{ij} + c\operatorname{tr}(\dot{\p}) \right)(\beta - \theta) \\
(B) - (B') &= (\varphi - \varphi^{ij}h_{ij}) |\nabla\varphi|^{2} + 2\varphi^{ij}h^{k}_{i}\nabla_k \varphi \nabla_j \varphi \\ 
    &\quad - (\varphi - \varphi^{ij}h_{ij})|\nabla\varphi|^{2} - 2\varphi^{ij}h^{k}_{i}\nabla_k\varphi\nabla_j\varphi \\    
        &= 0 \\
(C) - (C') &= \varphi^{ij,kl} \a_{ij} \a_{kl} + \varphi^{kl,ij} (\gamma_{ij}\gamma_{kl}  - 2\a_{ij} \gamma_{kl})\\ 
    &= \varphi^{ij,kl} (\a_{ij} - \gamma_{ij}) (\a_{kl} - \gamma_{kl}) \\
&= \varphi^{ij,kl} \eta_{ij} \eta_{kl} \\
(D) - (D') &= 2b^{il}\varphi^{jk} (2\nabla^2_{ij}\varphi\varphi(h^2)_{kl} + \varphi(h^2)_{ij}\varphi(h^2)_{kl}) \\
&\quad + 2b^{il} \varphi^{jk} \left(\gamma_{ij} \gamma_{kl} - 2\a_{ij} \gamma_{kl} + \nabla^2_{ij}\varphi\nabla^2_{kl}\varphi\right) \\
&= 2b^{il}\varphi^{jk} \big(\nabla^2_{ij}\varphi\nabla^2_{kl}\varphi +2\nabla^2_{ij}\varphi\varphi(h^2)_{kl} + \varphi(h^2)_{ij}\varphi(h^2)_{kl} \\
    &\qquad\qquad - 2 \a_{ij} \gamma_{kl} + \gamma_{ij} \gamma_{kl} \big) \\
&= 2b^{il}\varphi^{jk} \left(\a_{ij}\a_{kl} - 2 \a_{ij} \gamma_{kl} + \gamma_{ij} \gamma_{kl} \right) \\ 
    &= 2b^{il}\varphi^{jk} \eta_{ij} \eta_{kl} \\
(E) - (E') &= c(R_{\beta} - R_{\theta}) \\ 
    &= cR.\\
\end{align*}
Multiplying everything by \(t\) and adding the result to \cref{eq:deltchi1} and using the definitions of $\b$ and $\t$ gives the result.
\end{proof}
We need two more lemmas to obtain a Harnack inequality for $H^{p}$-flow with $0<p<1.$  We start by rewriting the term $R$ in the evolution of \(\chi_2\) when the speed is a function of the mean curvature.

\begin{lemma}\label{RSphere}
Suppose that $\p=\p(H).$ Then the term $R$ in the evolution equation of $\chi_2$ takes the form
\eq{\label{RSphere1}
R&=2n\fr{\p''\p}{\p'}\br{\Box\p + \p\p^{ij}(h^2)_{ij} - b^{ij}\nabla_i \p \nabla_j \p}-n\fr{\p''\p^2}{\p'}\p^{ij}(h^2)_{ij}+2\p^2\p'H\\
    &\hp{=}+n\br{2\fr{\p''}{\p'}-\fr{\p''^{2}\p}{\p'^{3}}+\fr{\p'''\p}{\p'^{2}}}\p^{ij}\nabla_{i}\p\nabla_j\p\\
 &\hp{=}+\br{\p'H+\p}b^{ir}b^{j}_{r}\nabla_i\p\nabla_j\p-2\p'b^{ij}\nabla_i\p\nabla_j\p.
}
\end{lemma}
\pf{
This computation is tedious but straightforward. The key point is to replace the terms containing second covariant derivatives of $\p,$ namely $\Box\operatorname{tr}(\dot{\p})$ and $\a_{ij},$ by a term involving $\Box\p,$ which can then be replaced by $\b-\t$ and some curvature terms.
}
To obtain a Harnack estimate for $H^{p}$-flow, we will have to handle the middle term in \cref{RSphere1}; this term does not always have the favorable positive sign. To this aim, it is useful to add an auxiliary function of the speed. Using Proposition \ref{thm:Evchi} and \cref{RSphere}, it is straightforward to obtain the following evolution equation for $\chi_3=\chi_2+t\zeta$, where $\zeta=\zeta(\p)$ is a function of $\p.$
\begin{lemma}\label{lemma : lem9}
The quantity
$\chi_3=\chi_2+t\zeta$
evolves according to
\eq{\label{Evchibar1}
\del_t \chi_3 -\Box\chi_3 &= \left(\fr{\b-\t}{\d\p} + \p^{ij}(h^2)_{ij} + c\operatorname{tr}(\dot{\p})\right)\chi_2+\zeta \\
&\hp{=} + t\left(\p^{ij,kl} + 2b^{il}\p^{jk} - \fr{\p^{ij}\p^{kl}}{\d\p}\right)\eta_{ij}\eta_{kl} \\
&\hp{=} + ct\Big\{2n\fr{\p''\p}{\p'}\br{\Box\p + \p\p^{ij}(h^2)_{ij} - b^{ij}\nabla_i \p \nabla_j \p}\\ 
&\hp{=+t}+\br{\zeta'-n\fr{\p''\p}{\p'}}\p^{ij}(h^2)_{ij}\p+ \zeta'\operatorname{tr}(\dot{\p})\p+ 2\p^2\p'H\\ 
&\hp{=+t}+\br{n\br{2\fr{\p''}{\p'}-\fr{\p''^{2}\p}{\p'^{3}}+\fr{\p'''\p}{\p'^{2}}}-\zeta''}\p^{ij}\nabla_{i}\p\nabla_j\p \\
&\hp{=+t} + \br{\p'H+\p}b^{ir}b^{j}_{r}\nabla_i\p\nabla_j\p-2\p'b^{ij}\nabla_i\p\nabla_j\p\Big\}.
}
\end{lemma}
Proposition \ref{thm:Evchi}, and Lemmas \ref{RSphere}, \ref{lemma : lem9}  enable us to get a strong Harnack estimate for $H^{p}$-flow; see Section \ref{sec:Harnack} and Theorem \ref{thm: main 1}. Due to the presence of $\Box\operatorname{tr}(\dot{\p})$ in $R$ given in Proposition \ref{thm:Evchi}, it is not clear to us whether $\chi_2$ would results in Harnack estimates for curvature flows other than $H^{p}$-flow. As it will be shown, by weakening $\chi_2$ to $\chi_1=\chi_2+tc\p\operatorname{tr}(\dot{\p})$,  we can obtain (weak) Harnack estimates for all 1-homogeneous convex speeds.
\begin{proposition}\label{WeakHarnackEv}
The quantity
$\chi_1=t(\del_t\p-\t)+\d\p$
satisfies the evolution equation
\eq{\label{WeakHarnackEv12}\del_t\chi_1-\Box\chi_1 &= \br{\frac{\b-\t}{\delta\varphi} + \varphi^{ij}(h^2)_{ij} + c\frac{\d-1}{\d} \operatorname{tr}(\dot{\p})}\chi_1\\ 
&\quad +\fr{c\operatorname{tr}(\dot{\p})\p}{\d}(tc\operatorname{tr}(\dot{\p})+2\d) \\
&\quad + t\p^{ij,kl}\br{\eta_{ij}+c\p g_{ij}}\br{\eta_{kl}+c\p g_{kl}}\\ &\quad +t\br{2b^{il}\p^{jk}-\fr{\p^{ij}\p^{kl}}{\d\p}}\eta_{ij}\eta_{kl} \\
&\quad + tc\left\{2\p^2\p^{ij}h_{ij}+\left[\br{\p^{ij}h_{ij}+\p}b^{ir}-2\p^{ir}\right]b^{j}_{r}\nabla_{i}\p\nabla_{j}\p\right\}.
}
\end{proposition}
\pf{
We simply use the evolution of $\chi_2,$ cf.~\eqref{thm:Evchi1}, and add the evolution of $tc\operatorname{tr}(\dot{\p})\p.$ This is
\eq{\label{WeakHarnackEv2}\br{\del_t-\Box}\br{tc\operatorname{tr}(\dot{\p})\p}&=tc\Big(\operatorname{tr}(\dot{\p})\p\p^{ij}(h^2)_{ij}+c\operatorname{tr}(\dot{\p})^2\p+\p\del_t\operatorname{tr}(\dot{\p})\\
                    &\hp{-tk\Big(}-\p\Box\operatorname{tr}(\dot{\p})-2\p^{ij}\nabla_{i}\p\nabla_{j}\operatorname{tr}(\dot{\p})\Big)+c\operatorname{tr}(\dot{\p})\p.
                            }
Noting that
$$\del_t\operatorname{tr}(\dot{\p})=\del_t\br{\p^{ij}g_{ij}}=\p^{ij,kl}\br{\a_{kl}+c\p g_{kl}}g_{ij}$$
and adding \eqref{WeakHarnackEv2} to the term $R$ in \eqref{thm:Evchi1}, we obtain the claim.
}

\section{Harnack Inequalities}
\label{sec:Harnack}

In Euclidean space we recover differential Harnack inequalities for various speeds already discussed in \cite[Corollary 5.11 (1)]{Andrews:09/1994}, as can be seen by evaluating the evolution equation \eqref{thm:Evchi1} with $c=0.$

\begin{remark}\label{HarnackEucCor}
 Let $x$ be a strictly convex solution of \eqref{eq:CurvFlow} in Euclidean space. If $f$ is a $1$-homogeneous and inverse concave curvature function, then for contracting flows the pair
$$\p(f)=f^{\a},\quad 0<\a<\8, \quad \d= \fr{\a}{\a+1},$$
and for expanding flows the pair
$$\p(f)=-f^{-\b}, \quad 0<\b<1,\quad \d= \fr{\b}{\b-1}$$
satisfy $$\partial_t \p-b^{ij}\nabla_i\p\nabla_j\p+\frac{\d\p}{t}>0.$$
%If $f$ is inverse convex (e.q., the inequality (\ref{eq: inverse concavity}) is reversed), then for expanding flows the pair
%\eq{\p(f)=-f^{-\b}, \quad \b>1,\quad \d\geq \fr{\b}{\b-1}}
%satisfy $\chi<0.$
Compare with \cite[Theorem 5.6, Corollary 5.11]{MR1296393}. To the best of our knowledge, this is the first time such general Harnack inequalities (even in Euclidean space) have been proven in such generality by direct computation.
\end{remark}
Let us consider the spherical case.
\begin{theorem}
Suppose $f$ is a  convex curvature function and $\p=f$, then for $t>0$ we have
$$\partial_t \p-b^{ij}\nabla_i\p\nabla_j\p+\frac{1}{2}\frac{\p}{t}>0.$$
\end{theorem}
\begin{proof}
In view of the maximum principle and that $\chi_1$ is manifestly positive at $t=0$, it suffices to show that the right-hand side of (\ref{WeakHarnackEv12}) is positive whenever at some point in space-time $\chi_1=0$. By convexity of $\p$, the term with the coefficient $\p^{ij,kl}$ is positive. On the other hand, note that any 1-homogeneous curvature function $\p$ satisfies $\p b^{ij}\geq \p^{ij}.$ Therefore
$$\left(\br{\p^{ij}h_{ij}+\p}b^{ir}-2\p^{ir}\right)b^{j}_{r}\nabla_{i}\p\nabla_{j}\p\ge 0.$$ To complete the proof note that
$$\br{b^{il}\p^{jk}-\fr{\p^{ij}\p^{kl}}{\p}}\eta_{ij}\eta_{kl}\geq 0,$$
see the proof of \cite[Theorem 2.3]{Andrews:/2007} for example.
\end{proof}

Employing the evolution \cref{Evchibar1}, we can obtain a stronger Harnack inequality for the speed \(\p = H^{p}\) with \(p \in (0,1)\); the case $p=1$ was considered in \cite{BryanIvaki:08/2015}.
\begin{theorem} \label{thm: main 1}
If $\frac{1}{2}+\frac{1}{2n}\leq {p}< 1,$ then
\[
\partial_t H^{p} - b^{ij}\nabla_iH^{p}\nabla_jH^{p} - \frac{c {p}}{2{p}-1}H^{2{p}-1} + \frac{{p}}{{p}+1} \frac{H^{p}}{t} > 0.
\]
If $0<{p}\leq \frac{1}{2} + \frac{1}{2n}$ or $p=1$, then

\[
\partial_t H^{p} - b^{ij}\nabla_iH^{p}\nabla_jH^{p} - c n{p}H^{2{p}-1} + \frac{{p}}{{p}+1} \frac{H^{p}}{t} > 0.
\]
\end{theorem}
\begin{proof}
In order to prove Theorem \ref{thm: main 1}, we need to show that for
$$\p=H^{p},\quad \d=\fr{p}{p+1},$$
the quantity $\chi_3$ preserves its positivity at all $t>0.$ Here $\zeta$ is chosen to be
$$\zeta(\p)=\begin{cases} 0, &0<p\leq \fr{1}{2}+\fr{1}{2n}\\
                    p\br{n-\fr{1}{2p-1}}\p^{2-\fr{1}{p}}, &\fr{1}{2}+\fr{1}{2n}<p< 1.\end{cases}$$
However, to avoid confusion, we will keep the general form as long as possible.
At time $t=0,$ $\chi_3$ is positive.
%%%%%%%%%%%% We need to impose strict mean convexity in the main theorem here!!!!!!! %%%%%%
Thus suppose there exists a first time $t_0$ and a point $x_0$ in $M_{t_0},$ such that $\chi_3(t_0,x_0)=0.$
Then we also obtain
$$\chi_2=-t\zeta\Rightarrow \b-\t=-\fr{\d\p}{t}-\zeta.$$
Thus, using \eqref{Evchibar1}, we obtain at $(t_0,x_0):$
\eq{0&\geq \del_t\chi_3-\Box\chi_3\\
        &=2\zeta-2n\d\fr{\p''\p^2}{\p'}+t\br{\p^{ij,kl}+2b^{il}\p^{jk}-\fr{\p^{ij}\p^{kl}}{\d\p}}\eta_{ij}\eta_{kl}\\
        &\hp{=}+t\Big\{\fr{\zeta^2}{\d\p}-2n\fr{\p''\p}{\p'}\zeta+2\p^2\p'H+\br{\zeta'\p-\zeta}\p^{ij}g_{ij}    \\
        &\hp{=+t} +\br{\zeta'\p-n\fr{\p''\p^2}{\p'}-\zeta}\p^{ij}(h^2)_{ij}+\br{\p'H+\p}b^{ir}b^j_r\nabla_i\p\nabla_j\p\\
        &\hp{=+t}-2\p'b^{ij}\nabla_i\p\nabla_j\p+\br{n\br{2\fr{\p''}{\p'}-\fr{\p''^{2}\p}{\p'^{3}}+\fr{\p'''\p}{\p'^{2}}}-\zeta''}\p^{ij}\nabla_{i}\p\nabla_j\p\Big\}\\
        &\geq2\zeta-2n\d\fr{\p''\p^2}{\p'}+t\br{\p^{ij,kl}+2b^{il}\p^{jk}-\fr{\p^{ij}\p^{kl}}{\d\p}}\eta_{ij}\eta_{kl}\\
        &\hp{=}+t\Big\{\fr{\zeta^2}{\d\p}+2\p^2\p'H-2n\fr{\p''\p}{\p'}\zeta+n\br{\zeta'\p-\zeta}\p'    \\
        &\hp{=+t} +\br{\zeta'\p-n\fr{\p''\p^2}{\p'}-\zeta}\p^{ij}(h^2)_{ij}\\
        &\hp{=+t}+\br{n\br{2\fr{\p''}{\p'}-\fr{\p''^{2}\p}{\p'^{3}}+\fr{\p'''\p}{\p'^{2}}}-\zeta''+\fr{\p}{\p'H^2}-\fr{1}{H}}\p^{ij}\nabla_{i}\p\nabla_j\p\Big\},}

where in the last inequality we used
$$\p-\p'H=(1-p)H^p\geq 0$$
and
$$Hb^{ij}\geq g^{ij}$$
in the sense of bilinear forms.

To finish the proof, we need to show that the right-hand side is positive. If $\zeta=0$, this is straightforward:
\[\varphi''<0,\quad n\br{2\fr{\p''}{\p'}-\fr{\p''^{2}\p}{\p'^{3}}+\fr{\p'''\p}{\p'^{2}}}+\fr{\p}{\p'H^2}-\fr{1}{H}\geq 0.\]
For the second case that $\zeta\neq 0$, note that
$$2\zeta-2n\d\fr{\p''\p^2}{\p'}\geq 0\quad\mbox{for}\quad p\geq \frac{n+1}{2n},$$
$$\br{\p^{ij,kl}+2b^{il}\p^{jk}-\fr{\p^{ij}\p^{kl}}{\d\p}}\eta_{ij}\eta_{kl}\geq0\quad\mbox{for}\quad\delta=\frac{p}{p+1},$$
$$\fr{\zeta^2}{\d\p}-2n\fr{\p''\p}{\p'}\zeta+n\br{\zeta'\p-\zeta}\p'\geq 0\quad\mbox{for}\quad p\geq\frac{n+1}{2n},~\delta=\frac{p}{p+1},$$
$$\zeta'\p-n\fr{\p''\p^2}{\p'}-\zeta\geq 0\quad\mbox{for}\quad\frac{n+1}{2n}\leq p\leq 1,$$
$$n\br{2\fr{\p''}{\p'}-\fr{\p''^{2}\p}{\p'^{3}}+\fr{\p'''\p}{\p'^{2}}}-\zeta''+\fr{\p}{\p'H^2}-\fr{1}{H}=0.$$
\end{proof}

\section{Preserving convexity}
\label{sec:convexity} 

In the derivation of the Harnack inequalities we have assumed the strict convexity of flow hypersurfaces. In this section, we show that strict convexity is preserved for all flows in the sphere for which we could prove the Harnack inequality. In Euclidean space, the question of preserved convexity has been addressed more thoroughly. It is also known that there is a variety of examples where convexity is lost for contracting flows \cite{AndrewsMcCoyZheng:07/2013}, the authors also discuss necessary and sufficient conditions for preserving convexity. In other special situations preserved convexity was proved, e.g., see \cite{Andrews:/1994b}, \cite{Andrews:/2007}, \cite{Andrews:04/2010} and \cite{Schulze:/2005}.
\begin{proposition}
 Let $M_{0}$ be a closed and strictly convex initial hypersurface for the curvature flow equation
(\ref{eq:CurvFlow}). Suppose either $\p=f$ with $f$ as a convex curvature function, or  $\p=H^{p}$ and $p\in (0,1)$.
Then all flow hypersurfaces are strictly convex.
\end{proposition}
\pf{
Let us first treat the case $\p=H^{p}$ and $0<p<1.$ The inverse curvature function of $H$ is the harmonic mean curvature
\eq{\~H=\br{\sum_{i=1}^n\k_i^{-1}}^{-1}.}
Let $T$ be the first time, where the strict convexity is lost. Then on the time interval $[0,T)$ the dual flow defined via the Gauss map is well defined and reads
\eq{\label{DualFlow}\dot{\~x}=\fr{1}{\~H^{p}}\~\nu,}
where the harmonic mean curvature $\~H$ is now evaluated at $\~\k_i=\k_i^{-1},$ compare \cite{Gerhardt:/2015} for the derivation of the dual flow. As a curvature function, the harmonic mean curvature is 1-homogeneous, strictly monotone, concave and vanishes on the boundary of $\G_{+} $ (regarding the concavity see \cite[Lemma~2.2.12, Lemma~2.2.14]{Gerhardt:/2006}). For flows of the kind \eqref{DualFlow} uniform curvature estimates were deduced in \cite[Lemma~4.7]{MakowskiScheuer:/2013}, implying that the $\~\k_{i}$ are bounded. This means that up to time $T$ uniform convexity is preserved for the original flow, which contradicts the definition of $T,$ if $T$ is not the collapsing time.

Now we treat convex speeds. We use Hamilton's maximum principle for tensors to deduce that the tensor
$S_{ij}=h_{ij}-\e g_{ij}$
remains non-negative, if $\e$ is chosen so that this is the case initially. We use the evolution equations \eqref{eq:delt_metric} and \eqref{eq:delt_sff_box} to deduce
\eq{\label{PresConv}\del_tS_{ij}-\Box S_{ij}&=\p'f^{kl}(h^2)_{kl}h_{ij}-(\p'f+\p)(h^2)_{ij}+c(\p'f+\p)g_{ij}\\
                    &\hp{=}-c\p'f^{kl}g_{kl}h_{ij}+2\e \p h_{ij}+\p^{kl,rs}\nabla_{i}h_{kl}\nabla_{j}h_{rs}\\
                    &=:N_{ij}.}
Using convexity of $f$, for any unit length null eigenvector $\eta$ of $S_{ij}$ we have
\eq{N_{ij}\eta^i\eta^j\geq -2\e^2f+2cf-\e cf^{kl}g_{kl}+2\e^2 f.}
On the other hand, for convex and $1$-homogeneous curvature functions there holds
$f^{kl}g_{kl}\leq n,$
cf.~\cite[Lemma~2.2.19]{Gerhardt:/2006}. Furthermore, $S_{ij}$ is still positive semi-definite and thus $f\geq n\e.$
Hence
$N_{ij}\eta^i\eta^j\geq n\e>0.$
}



\bibliographystyle{amsplain}
\bibliography{Bibliography.bib}


\end{document} 
